\documentclass{article}
\usepackage{graphicx}
\usepackage{makecell}
\usepackage[T1]{fontenc}
\usepackage{minted}
%\usepackage{tgpagella}      
\usepackage[dvipsnames]{xcolor}
\usepackage{booktabs}
\usepackage{longtable}
\usepackage{tgpagella} 
\usepackage{colortbl}
\usepackage[most]{tcolorbox}
\RequirePackage{fontawesome}
\usepackage{hyperref}
\hypersetup{
    colorlinks=true, 
    linktoc=all,     
    linkcolor=black!80,
}
\usepackage{geometry}
\geometry{a4paper, total={5.5in, 9in}}
\renewcommand{\arraystretch}{1.4}
\newcommand{\must}{\cellcolor{Green}{M}}
\newcommand{\should}{\cellcolor{LimeGreen}{S}}
\newcommand{\could}{\cellcolor{RedOrange}{C}}
\newcommand{\wont}{\cellcolor{BrickRed}{W}}

\tcbset{%
    theoLeft/.style={%
        enhanced,
        breakable,
        sharp corners,
        toprule=1mm, rightrule=0pt, bottomrule=1mm, leftrule=0pt,
        colback=#1!5, colframe=white, coltitle=#1!80!black, 
        detach title,
        overlay unbroken and first ={
            \node[minimum width=1cm, anchor=north west, font=\bfseries] 
               at ([yshift={.4cm}]frame.north west) {\tcbtitle};
        }
    }
}

\definecolor{exampleBox}{RGB}{ 121, 54, 149  }

\newtcbtheorem[auto counter]{code}{}
{theoLeft=exampleBox}{cl}

\title{\huge Testing}
\author{Gabriele Chignoli}
\date{Luglio 2025}
\begin{document}
\setcounter{tocdepth}{5}
\maketitle
\tableofcontents
\newpage

\section{Introduzione}
Nel seguente documento vengono mostrate le varie attività di test eseguito sul software Pantrymanager. Si intende documentare per ogni componente testata il nome e la posizione del componente, la data di esecuzione, il tipo e l'obiettivo del test ed i risultati ottenuti.

\section{Test Unitari}
\subsection{Costruttore classe \texttt{Product} - 25/07/2025}

\paragraph{Obiettivo}
Viene testato il costruttore di oggetti \textit{Product}
\begin{code*}{}
\begin{minted}{java}
public Product(String name, float weight, int qty, 
    float calories, LocalDate expiration_date) {
	... 
}
    \end{minted}
\end{code*}

\paragraph{Input}
Sono stati utilizzati 5 valori differenti per testare il nome del prodotto, 5 per il peso, e solamente 1 per gli altri attributi. Sono stati inclusi anche valori che non hanno senso per quello che rappresentano, ma verranno vincolati più avanti e ad un livello superiore della struttura, verso l'interfaccia utente.  

\paragraph{Aspettative e risultati}
Il test era molto semplice e testava una parte molto piccola di codice che difficilmente risulta problematica. L'obiettivo era prendere confidenza con JUnit, per poi provare a seguire in futuro anche un approccio guidato dei test. 

I risultati del test vengono caricati attraverso un file xml esportato direttamente da Eclipse. 

\paragraph{Directory Componente} 
\begin{code*}
    .../src/main/java/controller/Product.java
\end{code*}
\paragraph{Directory Test} 
\begin{code*}
    .../src/test/java/controller/ProductTest.java
\end{code*}

\subsection{\texttt{ProductManager} - 26/08/2025}
\paragraph{Obiettivo} Viene testata la classe \texttt{ProductManager} che effettua le operazioni sul DataBase, in particolare i seguenti metodi: 
\begin{code*}{}
\begin{minted}{java}
public static int saveProduct(Product to_save) {
    return run(OperationMode.SAVE, to_save);	
}

public static int deleteProduct(String name) {
    ...
    return run(OperationMode.DELETE, to_delete);
}

public static int modifyProduct(Product p) {
    return run(OperationMode.MODIFY, p);
}

public static List<Product> getProducts() {
	...
    return entity_manager.createQuery(criteria).getResultList();
}

\end{minted}
\end{code*}

\paragraph{Input} Sono stati utilizzati diversi prodotti, per i quali viene verificato che le operazioni abbiano correttamente salvato, eliminato o modificato nel database \texttt{PANTRY.mv.db}. 

\paragraph{Aspettative e risultati}
Il test voleva verificare principalmente una corretta interazione con il database, per la quale si considera il test superato; tuttavia ha anche evidenziato come gli input vengano semplicemente salvati senza essere controllati. In particolare, a parte il nome che non può essere omesso e che deve essere unico, gli altri parametri numerici possono assumere valori negativi o comunque non voluti se si agisce a questo livello. 

Passando per l'interfaccia grafica questo problema viene evitato non permettendo all'utente di inserire certi valori; rimane da decidere se introdurre un ulteriore verifica a questo stadio. 

\paragraph{Directory Componente} 
\begin{code*}
    .../src/main/java/controller/ProductManager.java
\end{code*}
\paragraph{Directory Test} 
\begin{code*}
    .../src/test/java/controller/ProductManagerTest.java
\end{code*}

\subsection{\texttt{MainWindow} - 28/08/2025}
\paragraph{Obiettivo} Viene testata la classe \texttt{MainWindow} che gestisce l'interfaccia grafica. Purtroppo la classe non risulta facilmente testabile: per i metodi getter l'operazione è immediata, mentre per verificare che l'interfaccia grafica si comporti correttamente si è stati in grado solo di verificare che venisse instanziata la classe e fossero ritornati i valori attesi.  

\paragraph{Input} Non sono stati utilizzati input ma si è verificato che la classe istanziasse o non istanziasse (ritornasse null) per i metodi getter. Si consiglia la visione dei file su \textit{Github}. 

\paragraph{Aspettative e risultati}
Il test dell'interfaccia rimane un'attività da approfondire, poiché, pur essendo stato verificato durante lo sviluppo che gli input fossero salvati e modificati correttamente utilizzando l'interazione e verifica manuale dello sviluppatore, sarebbe più opportuno automatizzare il processo. 

Una delle possibili soluzione sarebbe predisporre la classe \texttt{MainWindow} di metodi che simulino l'input dell'utente modificando i campi senza passare per l'interfaccia, ma agendo sui componenti direttamente. Poi attraverso una classe esterna eseguire i metodi e verificarne l'output. 

\paragraph{Directory Componente} 
\begin{code*}
    .../src/main/java/view/MainWindow.java
\end{code*}
\paragraph{Directory Test} 
\begin{code*}
    .../src/test/java/view/MainWindow.java
\end{code*}

\subsection{Test di Sistema}
Risultati di tutti i test eseguiti insieme. 
\begin{figure}[H]
    \centering
    \includegraphics[width=.9\linewidth]{imgs/test_result.png}
    \caption{System Test results}
    \label{fig:placeholder}
\end{figure}

\subsection{Test di Copertura}
Viene mostrato quanto codice i test coprono, ovvero quante righe di codice eseguono (raggiungono) e testano.
\begin{figure}[H]
    \centering
    \includegraphics[width=\linewidth]{imgs/coverage.png}
    \caption{Test Coverage}
    \label{fig:placeholder}
\end{figure}
\paragraph{Considerazioni}
\begin{itemize}
    \item Come si può vedere non si è riusciti a testare le azioni del \texttt{controller}, legate alla interfaccia utente.
    \item Le classi legate a \texttt{dish} non sono utilizzate nella versione rilasciata e quindi non sono state testate.
    \item \texttt{MainWindow} ha una buona copertura ma non si ritiene comunque di aver testato al meglio la classe.
\end{itemize}




\end{document}
